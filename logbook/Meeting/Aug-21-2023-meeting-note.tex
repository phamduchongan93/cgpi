\documentclass{beamer}
\usepackage{graphicx}
\usepackage{hyperref}
\title{CPP Community Garden Meeting Note}
\author{An Pham, Myriam}
\institute{Calpoly Pomona}
\date{\today}
\usetheme{Boadilla}
%\usetheme{Berkeley}
\usepackage{graphicx}
\usepackage{booktabs}
\usepackage{hyperref}
\usepackage[english]{babel}

\begin{document}
\frame{\titlepage}

\begin{frame}
\frametitle{Overview}
\begin{enumerate}
  \item Technical Diffculty of Old Sensors.
  \item Establish the goals of the object that may benefit the campus.
  \item Broadcast LORA signal.
  \item Bill of Material for lab environment.
  \item Revisit the Cloud integration for LORA sensors. 
   
 \end{enumerate}
\end{frame}

\begin{frame}
  \frametitle{Wired version }
  \begin{itemize}
    \item Most of Arduino Wire version requrie exposing the wire connector. 
    \item Can be used indoor environment.
    \item The effective of the sensors relies on the wireless communication. 
      \begin{itemize}
        \item Reliable connection helps to reduce the cost of storing which add power consumption. 
	\item Pick the right signal sending cycle will help to improve the effectiveness of the sensors. 
      \end{itemize}
  \end{itemize}
 \end{frame}
 
 \begin{frame}[t]
   \frametitle{Technology of Moisture Sensors}
   \framesubtitle{subtitle}
   \begin{itemize}
     \item There are two type of technology for moisture sensors 
       \begin{itemize}
         \item Time-Domain Reflectometry (TDR probes). Two connectors mounted to the ground, It detects the water condensation in the ground. 
	 \item Capacitance (C-probes, Frequency-Domain Reflectometers). 
	 \item Theses two technology require professional installation. TDR, 
	   \item FDR and C-Probes have all worked well, but have theirlimitations. They read only a small volume of soil surrounding the guides or probes. FDR and C-Probes are also sensitive to air gaps between the access tube and the soil (source: College of Agriculture and Life Science - Methods of Measuring for Irrigation Scheduling)
	   \item the ECOWITT soil moisture sensor used the Capacitance.
       \end{itemize}
   \end{itemize}
 \end{frame}

\begin{frame}
  \frametitle{ECOWITT Sensors Kit}
  \begin{itemize}
    \item This 8 sensors kit can operate near the building where wifi access is available.
    \item Try to raise the height of solarbox, this would improve the LORAWAN as well.  
    \item Agriculture water problems deprive from weather and soil moisture. 
  \end{itemize}
\end{frame}

\begin{frame}
  \frametitle{Both Lora and None Lora Deployment}
  \begin{itemize}
    \item Understand the correlation of bandwith and power consumption.
      \begin{itemize}
        \item sensors are end devices in Lora network.
	\item reduce the transmitting size would reduce power consumption. 
	\item how long does it take to send one message of LoraWAN?
	\item Seperate the Sensors Network from the could. The Cloud shoud be last mileage component.
      \end{itemize}
  \end{itemize}
\end{frame}


\begin{frame}[t]
  \frametitle{Agenda}
 \begin{itemize}
   \item Discuss with Melvin and Myriam regarding DIY gateway solution and Implementing AWS IoT Core
     \begin{itemize}
       \item IoT can aggregate the signal sent from devices. 
       \item Address the previous of Melvin's Research on Greengrass. 
	 \begin{itemize}
	   \item What features we need on Greengrass?
	   \item Can the functions performed remotely?
	 \end{itemize}
     \end{itemize}
   \item Decide the prototype for Product Demo in Fall 2023 presentation.
     \begin{itemize}
       \item Outcome: Able to set up LORA soil moisture sensors along with AWS iot Core.
     \end{itemize}
 \end{itemize} 
\end{frame}


\begin{frame}[t]
  \frametitle{Reference }
  \framesubtitle{Sources}
 \begin{itemize}
    \item \url{https://docs.aws.amazon.com/iot/latest/developerguide/connect-to-iot.html} {LORA and Non-LORA AWS IoT Core}.
    \item College of Agrigulture and Life Science.
  \end{itemize}
  
  
\end{frame}


\end{document}
